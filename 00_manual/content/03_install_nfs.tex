\section{Instalación y configuración de servidor NFS}


NFS (sistema de archivos de red: «Network File System») es un protocolo que permite acceso remoto a un sistema de archivos a través de la red. Todos los sistemas Unix pueden trabajar con este protocolo.

% Para el montaje de este servidor en las máquinas con debian y arch linux se basa en la siguiente información

% \begin{itemize}
%     \item \url{http://l.github.io/debian-handbook/html/es-ES/sect.nfs-file-server.html}
    
%     \item \url{https://wiki.archlinux.org/index.php/NFS}      
    
    
% \end{itemize}

\subsection{Configuración servidor}

\begin{enumerate}

    \item En el master
    
\begin{lstlisting}[language=bash,style=mystyle] 
$ sudo apt install nfs-kernel-server portmap
$ mkdir sharedFolder
$ ls sharedFolder/ 
\end{lstlisting}

    \item Cambiar el propietario de la carpeta a nadie.
    
\begin{lstlisting}[language=bash,style=mystyle]    
$ sudo chown nobody:nogroup sharedFolder/
\end{lstlisting} 


    \item Ver el estado del servidor:
    
\begin{lstlisting}[language=bash,style=mystyle]    
$ sudo /etc/init.d/nfs-kernel-server status
\end{lstlisting} 


    \item Detener el servicio:
    
\begin{lstlisting}[language=bash,style=mystyle]    
$ sudo /etc/init.d/nfs-kernel-server stop
\end{lstlisting} 


    \item Modificar el archivo exports:
    
\begin{lstlisting}[language=bash,style=mystyle]    
$ sudo pico /etc/exports
\end{lstlisting} 


    \item Añadir:
    
\begin{lstlisting}[language=bash,style=mystyle]    
/home/gfnun/sharedFolder/ *(rw,sync,no_subtree_check)
/home/gfnun/Geant4-10.6/ *(rw,sync,no_subtree_check)
/home/gfnun/Geant4-10.5/ *(rw,sync,no_subtree_check)
\end{lstlisting} 



    \item Volver a la terminal, y hacer que todos los directorios colocados allí se exporten:
    
\begin{lstlisting}[language=bash,style=mystyle]    
$ sudo exportfs -a
\end{lstlisting} 


    \item Verificar:
    
\begin{lstlisting}[language=bash,style=mystyle]    
$ sudo exportfs
\end{lstlisting} 


    \item Salida en la terminal:
    
\begin{lstlisting}[language=bash,style=mystyle]    
/home/gfnun/sharedFolder
		<world>
/home/gfnun/Geant4-10.6
		<world>
/home/gfnun/Geant4-10.5
		<world>
\end{lstlisting} 


    \item En los esclavos:
    
\begin{lstlisting}[language=bash,style=mystyle]    
$ sudo apt-get install nfs-common portmap
$ mkdir -p sharedFolder/
$ ls sharedFolder
$ mkdir -p Geant4-10.6/
$ ls sharedFolder
\end{lstlisting} 


    \item En el maestro iniciar el servidor:
    
\begin{lstlisting}[language=bash,style=mystyle]    
$ sudo /etc/init.d/nfs-kernel-server start
\end{lstlisting} 


    \item Ver el estado:
    
\begin{lstlisting}[language=bash,style=mystyle]    
$ sudo /etc/init.d/nfs-kernel-server status
\end{lstlisting} 

    \item Ahora compartir las carpetas en cada esclavo, por ejemplo para el esclavo 1 (\emph{slave1}):
    
\begin{lstlisting}[language=bash,style=mystyle] 
$ ssh slave1
$ sudo mount master:/home/gfnun/sharedFolder sharedFolder/
$ sudo mount master:/home/gfnun/Geant4-10.6/ Geant4-10.6/
$ sudo mount master:/home/gfnun/Geant4-10.5/ Geant4-10.5/
\end{lstlisting} 


    \item Verificar que se esta compartiendo:
    
\begin{lstlisting}[language=bash,style=mystyle]    
$ df -h
\end{lstlisting}


    \item La salida debería ser:
    
\begin{lstlisting}[language=bash,style=mystyle]    
master:/home/gfnun/sharedFolder               723G  440G  247G  65% /home/gfnun/sharedFolder
master:/home/gfnun/Geant4-10.6                723G  440G  247G  65% /home/gfnun/Geant4-10.6
master:/home/gfnun/Geant4-10.6                723G  440G  247G  65% /home/gfnun/Geant4-10.5
\end{lstlisting} 


    \item En uno de los esclavos podemos probar que se comparten los archivos:
    
\begin{lstlisting}[language=bash,style=mystyle]    
$ cd sharedFolder/
$ touch test.txt
\end{lstlisting} 

y verificar que se vea desde el maestro.


\end{enumerate}



\newpage