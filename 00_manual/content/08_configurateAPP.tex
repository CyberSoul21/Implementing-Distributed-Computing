\section{Configurar una aplicación para uso con MPI}

Para este ejemplo se va usar la aplicación g4pntest, ubicada en este repositorio en la carpeta 00_g4_apps/

\subsection{Configuración sesión con G4mpi}

\begin{enumerate}
    \item Agregar las librerias necesarias para el uso de G4mpi, en PNtest.cc:

\begin{lstlisting}[language=bash,style=mystyle]
#include "G4MPImanager.hh"
#include "G4MPIsession.hh"
\end{lstlisting}
    
    \item Luego en la función main() del archivo PNtest.cc se añade las siguientes líneas de código, para este ejemplo se añade después de definir la semilla:

\begin{lstlisting}[language=bash,style=mystyle]
  // --------------------------------------------------------------------
  // MPI session
  // --------------------------------------------------------------------
  G4MPImanager* g4MPI = new G4MPImanager(argc, argv);
  // MPI session (G4MPIsession) instead of G4UIterminal
  // Terminal availability depends on your MPI implementation.
  G4MPIsession* session = g4MPI-> GetMPIsession();
  G4String prompt = "";
  prompt += "G4MPI";
  prompt += "(%s)[%/]:";
  session-> SetPrompt(prompt);
\end{lstlisting}
    
    \item En este caso la aplicación tiene activado Multithreading (MT), se debe dejar sin MT:
\begin{lstlisting}[language=bash,style=mystyle]
//#ifdef G4MULTITHREADED //Modified vy Javier
//  G4MTRunManager* runManager = new G4MTRunManager;
//  runManager-> SetNumberOfThreads(4);
//#else
  G4RunManager* runManager = new G4RunManager;
//#endif
\end{lstlisting}    
    
    \item Para la sesión se comentan o se quitan las siguientes líneas de código:
\begin{lstlisting}[language=bash,style=mystyle]
/* ORIGINAL
  // get the pointer to the User Interface manager
  G4UImanager* UI = G4UImanager::GetUIpointer();  

  if(argc == 1)
    {
#ifdef G4UI_USE
      G4UIExecutive* ui = new G4UIExecutive(argc, argv);
#ifdef G4VIS_USE
      UI->ApplyCommand("/control/execute vis.mac");
#endif
      ui->SessionStart();
      delete ui;
#endif
    }
  // Batch mode
  else
    {
      G4String command = "/control/execute ";
      G4String fileName = argv[1];
      UI->ApplyCommand(command+fileName);
    }
*/
\end{lstlisting}
    
    \item Se inicia la sesión:
\begin{lstlisting}[language=bash,style=mystyle]
session-> SessionStart();
\end{lstlisting}

    \item Por último se elimina la sesión de mpi agregando la siguiente línea de código antes del return 0:   
\begin{lstlisting}[language=bash,style=mystyle]
  delete g4MPI;
  delete runManager;
  return 0;
\end{lstlisting}    
\end{enumerate}



\subsection{Configuración CMakeLists.txt}


\begin{enumerate}

    \item Para el paquete de G4mpi agregar:
\begin{lstlisting}[language=bash,style=mystyle]
find_package(G4mpi REQUIRED) 
\end{lstlisting}

    \item Para incluir los directorios, se deja así:
\begin{lstlisting}[language=bash,style=mystyle]
include_directories(${CMAKE_CURRENT_SOURCE_DIR}/include
			      ${Geant4_INCLUDE_DIR}
			      ${G4mpi_INCLUDE_DIR})
\end{lstlisting}

    \item Añadir ejecutable y enlace con las librerias Geant4:
\begin{lstlisting}[language=bash,style=mystyle]
target_link_libraries(PNtest ${G4mpi_LIBRARIES})
\end{lstlisting}

\end{enumerate}



\subsection{Configuración archivos .csv}

\begin{enumerate}
    
    \item La función RunAction::BeginOfRunAction(const G4Run* aRun), se deja así, para la generación de cada archivo .csv en cada nodo o cada hilo:
\begin{lstlisting}[language=bash,style=mystyle]
void RunAction::BeginOfRunAction(const G4Run* aRun)
{
  G4cout << "### Run " << aRun->GetRunID() << " start." << G4endl;
  G4RunManager::GetRunManager()->SetRandomNumberStore(false);

  // ntuples will be written on each rank
  G4int rank = G4MPImanager::GetManager()->GetRank();
  std::ostringstream fname;
  fname<<"out_test-rank"<<rank;


  G4AnalysisManager* analysisManager = G4AnalysisManager::Instance();
  analysisManager->SetVerboseLevel(2);
  analysisManager->SetFileName(fname.str());

  analysisManager->OpenFile();
  analysisManager->CreateH1("Edep","Energy (keV)", 1000, 0., 1000.); // Id = 0
}
\end{lstlisting}


    \item Por último la función void RunAction::EndOfRunAction(const G4Run* aRun) se deja así:
    
\begin{lstlisting}[language=bash,style=mystyle]
void RunAction::EndOfRunAction(const G4Run* aRun)
{
  //if(!IsMaster()) return;
  //G4int nofEvents = aRun->GetNumberOfEvent();
  //if (nofEvents == 0) return;
  if(!IsMaster()) return;
  G4int nofEvents = aRun->GetNumberOfEvent();
  if (nofEvents == 0) return;

  G4cout << "RunActionMaster::EndOfRunAction" << G4endl;
  const G4int rank = G4MPImanager::GetManager()-> GetRank();
 
  G4cout << "=====================================================" << G4endl;
  G4cout << "Start EndOfRunAction for master thread in rank: " << rank<<G4endl;
  G4cout << "=====================================================" << G4endl;
 
  if ( !G4MPImanager::GetManager()->IsExtraWorker() ) {

    //Save histograms *before* MPI merging for rank #0
    if (rank == 0) {
      auto analysisManager = G4AnalysisManager::Instance();
      analysisManager->Write();
      analysisManager->CloseFile(false); // close without resetting histograms 
    }


      //Merge of g4analysis objects
      G4cout << "Go to merge histograms " << G4endl;
      G4MPIhistoMerger hm(G4AnalysisManager::Instance());
      hm.SetVerbosity(1);
      hm.Merge();
      G4cout << "Done merge histograms " << G4endl;
    }
 
   //Save g4analysis objects to a file
   //NB: It is important that the save is done *after* MPI-merging of histograms
   //One can save all ranks or just rank0, chane the if  
  if (true){  
  // if (rank == 0){
    auto analysisManager = G4AnalysisManager::Instance();
    if (rank == 0) {
      analysisManager->OpenFile("out_test-merged");  }
    analysisManager->Write();
    analysisManager->CloseFile();  }

   G4cout << "===================================================" << G4endl;
   G4cout << "End EndOfRunAction for master thread in rank: " << rank << G4endl;
   G4cout << "===================================================" << G4endl;


 // auto analysisManager = G4AnalysisManager::Instance();
 // analysisManager->Write();
 // analysisManager->CloseFile();
}
\end{lstlisting}



\end{enumerate}


\subsection{Configuración Clúster}

Para configurar el clúster y hacer uso de este con open mpi creamos un archio de texto plano llamado hostfile. Para este caso se tiene disponible 5 computadores con 8 hilos cada uno, es decir el clúster tiene 40 hilos disponibles. Si se desea usar todos los hilos disponibles el archivo debería erse así:

\begin{lstlisting}[language=bash,style=mystyle]
  master slots=8
  slave1 slots=8
  slave2 slots=8
  slave3 slots=8
  slave4 slots=8
\end{lstlisting}

Se puede administrar la cantidad de hilos a usar y en que computadores,por ejemplo:

\begin{lstlisting}[language=bash,style=mystyle]
  master slots=2
  slave1 slots=2
  slave2 slots=2
  slave3 slots=2
  slave4 slots=2
\end{lstlisting}

En este caso configuramos el \emph{clúster} para usar 2 hilos por cada computador lo que da un total de 10 hilos a usar por el \emph{clúster}.

\newpage

\subsection{Comando Ejecución con Open mpi}

A continuación el paso a paso para la ejecución de la aplicación:

Cargar variables de entorno:

\begin{lstlisting}[language=bash,style=mystyle]
$ source ~/Geant4-10.6/install/bin/geant4.sh
$ export PATH=/home/gfnun/sharedFolder/openmpi-install/bin:$PATH
$ export LD_LIBRARY_PATH=/home/gfnun/sharedFolder/openmpi-install/lib:$LD_LIBRARY_PATH
\end{lstlisting}


Compilar, para este caso por ejemplo el nombre de la carpeta donde se encuentra el proyecto es g4pntest_mpi:

\begin{lstlisting}[language=bash,style=mystyle]
$ cd g4pntest_mpi
$ mkdir build
$ cd build
$ cmake -DG4mpi_DIR=/home/gfnun/Geant4-10.6/geant4-mpi/lib/G4mpi-10.6.1 -DCMAKE_CXX_COMPILER=/home/gfnun/sharedFolder/openmpi-install/bin/mpicxx \
      –DGeant4_DIR=/home/gfnun/Geant4-10.6/install/lib/Geant4-10.6.1 ../
$ make -j`nproc`
\end{lstlisting}

Ejecución aplicación:

\begin{lstlisting}[language=bash,style=mystyle]
mpirun -np 40 -hostfile hostfile -x G4ABLADATA -x G4ENSDFSTATEDATA -x G4INCLDATA -x G4LEDATA -x G4LEVELGAMMADATA -x G4NEUTRONHPDATA -x G4PARTICLEXSDATA -x G4PIIDATA -x G4REALSURFACEDATA -x G4SAIDXSDATA -x G4RADIOACTIVEDATA -oversubscribe ./PNtest ../macros/THEmacro.mac
\end{lstlisting}

Si se requiere usar menos hilos, por ejemplo 12, el comando de ejecución sería así:

\begin{lstlisting}[language=bash,style=mystyle]
mpirun -np 12 -hostfile hostfile -x G4ABLADATA -x G4ENSDFSTATEDATA -x G4INCLDATA -x G4LEDATA -x G4LEVELGAMMADATA -x G4NEUTRONHPDATA -x G4PARTICLEXSDATA -x G4PIIDATA -x G4REALSURFACEDATA -x G4SAIDXSDATA -x G4RADIOACTIVEDATA -oversubscribe ./PNtest ../macros/THEmacro.mac
\end{lstlisting}


\newpage




