\section{Configuraciones básicas de cada equipo}

\subsection{Configuración de usuarios de cada máquina}
Todas las máquinas se configuran con el mismo nombre de usuario, así en cada máquina
abrimos una terminal y ejecutamos los siguientes comandos:

\begin{enumerate}
    \item Entrar al modo root:
    
\begin{lstlisting}[language=bash, style=mystyle]
$ ifconfig -a
\end{lstlisting} 


    \item Agregar el usuario:
    
\begin{lstlisting}[language=bash, style=mystyle]
$ adduser gfnun
\end{lstlisting} 

Se solicitara el ingreso de la clave y unos datos adicionales los cuales deberan
ingresarse si se desea, de lo contrario solo hacer \textit{enter} y continuar.
    

    \item Comprobar que se creo el usuario:
    
\begin{lstlisting}[language=bash, style=mystyle]
$ cd ~
$ ls /home
\end{lstlisting}

    \item Se debera ver la carpeta del usuario, adicional ingresamos para comprobar:
    
\begin{lstlisting}[language=bash, style=mystyle]
$ ssh gfnun@0.0.0.0
\end{lstlisting}

    \item Salir y se le da permisos de administrador, desde el root:
    
\begin{lstlisting}[language=bash, style=mystyle]
$ exit
$ adduser gfnun sudo
\end{lstlisting}
    
\end{enumerate}



\subsection{Configurar ip estática}

\begin{enumerate}
    
    \item Comprobar las interfaces de red que tiene el equipo.
    
\begin{lstlisting}[language=bash,style=mystyle]
$ ifconfig -a
\end{lstlisting}


    \item Edita el archivo de configuración de las interfaces de red con el siguiente comando:
    
\begin{lstlisting}[]
sudo nano /etc/network/interfaces
\end{lstlisting} 

    \item En este caso se va configurar la interfaz eth0,
    
\begin{lstlisting}[language=bash, style=mystyle]
# Configuración de dirección IP fija para el interfaz eth0
auto eth0
iface eth0 inet static
address 192.168.1.50
netmask 255.255.255.0
network 192.168.1.0
broadcast 192.168.1.255
gateway 192.168.1.1
\end{lstlisting} 


    \item Reiniciar las interfaces de red del equipo:
    
\begin{lstlisting}[language=bash,style=mystyle]
$ sudo /etc/init.d/networking restart
\end{lstlisting} 


    \item Alternativamente:
    
\begin{lstlisting}[language=bash,style=mystyle]
$ sudo ifconfig eth0 down 
$ sudo ifconfig eth0 up
\end{lstlisting} 

\end{enumerate}



\subsection{Configurar archivo \emph{Hosts}}

Configurar el archivo hosts con las ip de los PC's con los que se va hacer el \emph{cluśter} (ver sección configuraciones básicas de cada equipo, para configuración de ip estática). 

Para ingresar al archivo:

\begin{lstlisting}[language=bash,style=mystyle2]
$ sudo nano /etc/hosts
\end{lstlisting} 

El archivo \emph{hosts} debe quedar así. En este caso tenemos dos esclavos y un maestro.

\begin{lstlisting}[language=bash,caption={Archivo hosts en el maestro},style=mystyle2]
127.0.0.1       localhost
127.0.1.1       MrHerbert.gfnun.unal.edu.co     MrHerbert

#MPI Cluster Setup
192.168.9.144 master
192.168.9.135 slave1
192.168.9.109 slave2

# The following lines are desirable for IPv6 capable hosts
::1     localhost ip6-localhost ip6-loopback
ff02::1 ip6-allnodes
ff02::2 ip6-allrouters

\end{lstlisting} 


También se debe configurar el de cada esclavo así:

\begin{lstlisting}[language=bash,caption={Archivo hosts esclavo 1},style=mystyle2]
#MPI Cluster Setup
192.168.9.144 master
192.168.9.109 slave1

# The following lines are desirable for IPv6 capable hosts
::1     localhost ip6-localhost ip6-loopback
ff02::1 ip6-allnodes
ff02::2 ip6-allrouters

\end{lstlisting} 

Para el esclavo 2:

\begin{lstlisting}[language=bash,caption={Archivo esclavo 1},style=mystyle2]
#MPI Cluster Setup
192.168.9.144 master
192.168.9.109 slave2

# The following lines are desirable for IPv6 capable hosts
::1     localhost ip6-localhost ip6-loopback
ff02::1 ip6-allnodes
ff02::2 ip6-allrouters

\end{lstlisting} 



\subsection{Configurar ssh}
Lo siguiente es configurar los accesos a través de ssh entre las máquinas para que no
solicite clave cada vez que la máquina maestro (master) acceda a cada esclavo (slave1, slave2..etc)

\begin{enumerate}
    \item En el Master, generar la llave rsa:
    
\begin{lstlisting}[language=bash,style=mystyle]
$ ssh-keygen -t rsa
\end{lstlisting} 

    \item Verificar si se tiene el archivo authorized\_keys, hacer esto para cada nodo:
    
\begin{lstlisting}[language=bash,style=mystyle]
$ ls -l .ssh/
\end{lstlisting}

    \item Si no se encuentra, crearlo:
\begin{lstlisting}[language=bash,style=mystyle]
$ touch ~/.ssh/authorized_keys
\end{lstlisting} 

    \item Generar permisos al archivo:
\begin{lstlisting}[language=bash,style=mystyle]
$ chmod 600 ~/.ssh/authorized_keys
\end{lstlisting}

    \item Copiar llaves generadas desde el master a los esclavos:
\begin{lstlisting}[language=bash,style=mystyle]
$ scp .ssh/id_rsa.pub gfnun@slave1:.ssh/authorized_keys
$ scp .ssh/id_rsa.pub gfnun@slave2:.ssh/authorized_keys
\end{lstlisting}

    \item También se debe copiar en el mismo master para que no solicite clave al hacer ssh sobre el mismo:
\begin{lstlisting}[language=bash,style=mystyle]
$ cp ~/.ssh/id_rsa.pub ~/.ssh/authorized_keys
\end{lstlisting}

    \item Para el esclavo dos
\begin{lstlisting}[language=bash,style=mystyle]
$ scp .ssh/id_rsa.pub gfnun@slave2:.ssh/authorized_keys
\end{lstlisting} 

Esto es importante ya que con la implementación de mpi a veces se desea usar el mismo master para ejecutar aplicaciones.

    \item Así no pedirá contraseña cada vez que se va ingresar a cada nodo desde el master, bastara escribir algo así:
    
\begin{lstlisting}[language=bash,style=mystyle]
$ ssh slave2
\end{lstlisting} 
    
\end{enumerate}






 















\newpage

