\section{Instalación de Open-MPI}

Descargar open-mpi desde \url{https://www.open-mpi.org/software/ompi/v4.0/}, se descarga el archivo openmpi-4.0.3.tar.gz. Dejarlo en el directorio:

\begin{lstlisting}[language=bash,style=mystyle]    
$ pwd
/home/gfnun/
\end{lstlisting} 


\begin{enumerate}

    \item Descomprimir en la carpeta sharedFolder/ :
\begin{lstlisting}[language=bash,style=mystyle]    
$ cp openmpi-4.0.3.tar.gz sharedFolder/
$ cd sharedFolder/
$ mkdir openmpi-install/
$ tar -xzf openmpi-4.0.3.tar.gz   
\end{lstlisting}
 
    \item Compilar:
\begin{lstlisting}[language=bash,style=mystyle]  
$ cd openmpi-4.0.3/
$ mkdir build
$ cd build
$ ../configure --prefix=/home/gfnun/sharedFolder/openmpi-install/ --enable-mpi-cxx --enable-orterun-prefix-by-default 
$ sudo make -j4      
\end{lstlisting}
    
    \item Instalar:
\begin{lstlisting}[language=bash,style=mystyle] 
$ sudo make install
$ sudo ldconfig      
\end{lstlisting}
    
    \item Para usar open mpi instalado en la carpeta sharedFolder/ cargamos las variables de entorno:
\begin{lstlisting}[language=bash,style=mystyle]
$ export PATH=/home/gfnun/sharedFolder/openmpi-install/bin:$PATH
$ export LD_LIBRARY_PATH=/home/gfnun/sharedFolder/openmpi-install/lib:$LD_LIBRARY_PATH
\end{lstlisting}   
    
    \item Si se desea, se puede dejar en el bashrc:
\begin{lstlisting}[language=bash,style=mystyle]
echo "export PATH=/home/gfnun/sharedFolder/openmpi-install/bin:$PATH" >> $HOME/.bashrc
echo "export LD_LIBRARY_PATH=/home/gfnun/sharedFolder/openmpi-install/lib:$LD_LIBRARY_PATH" >> $HOME/.bashrc
\end{lstlisting}

así cargara automáticamente las variables de entorono de mpi cada vez que se inicie sesión.
    
\end{enumerate}



\subsection{Configurar Cluster}

Crear un archivo en donde vamos a listar los nodos del \emph{clúster} (nodos se refiere a cada máquina del \emph{clúster}). En este archivo se indica los procesos a usar por máquina. 

A continuación se presentan ejemplos del archivo: 

\begin{itemize}
    \item Creación del Archivo:
\begin{lstlisting}[language=bash,style=mystyle]
sudo touch my_hosts 
\end{lstlisting}


    \item Añadir las siguientes líneas, indicando el número de hilos a usar por cada nodo:
    
\begin{lstlisting}[language=bash,style=mystyle2]
master slots=1
slave1 slots=2
slave2 slots=2
\end{lstlisting}

    \item Ejecutar ejemplo:
\begin{lstlisting}[language=bash,style=mystyle2]
$ cd ..
$ cd examples
$ sudo mpicc hello_c.c -o hello_c
$ mpiexec -np 4 --hostfile my_hosts ./hello_cslave1
slave2
\end{lstlisting}

\end{itemize}



\subsection{Ejemplo de Comandos MPI}


\begin{lstlisting}[language=bash,style=mystyle2]
mpirun -np 6 --hostfile my_hosts openmpi-4.0.3/examples/hello_c | sort
mpirun -H slave1,slave2 openmpi-4.0.3/examples/hello_c | sort
mpirun -H master,slave1,slave2 openmpi-4.0.3/examples/hello_c | sort
mpirun -H master,slave1,slave2,slave2 openmpi-4.0.3/examples/hello_c | sort
mpirun -H master,slave1,slave2 -npernode 2 openmpi-4.0.3/examples/hello_c | sort
\end{lstlisting} 


\newpage