\section{Ejecutando ejemplos de geant4 en el sistema distribuido}

\subsection{Ejemplo exMPI01}

En esta sección se va ejecutar el ejemplo exMP01, ubicado en:

\begin{lstlisting}[language=bash,style=mystyle]
cd /home/gfnun/geant4.10.06/src/geant4.10.06.p01/examples/extended/parallel/MPI/examples/exMPI01
\end{lstlisting}

La documentación nos indica como debemos compilar el ejemplo.
\begin{lstlisting}[language=bash,style=mystyle]
mkdir build
cd build
cmake -DG4mpi_DIR=<where-G4mpi-wasintalled>/lib[64]/G4mpi -DCMAKE_CXX_COMPILER=mpicxx \
      -DGeant4_DIR=<your Geant4 install path>/lib[64]/Geant4-V.m.n <path-to-source>
      (V.m.n is the version of Geant4, eg. Geant4-9.6.0)
make
make install
\end{lstlisting}

Para este caso el ejemplo se compila así:

\begin{lstlisting}[language=bash,style=mystyle]
$ mkdir build
$ cd build
$ sudo cmake -DG4mpi_DIR=/home/gfnun/Geant4-10.6/geant4-mpi/lib/G4mpi-10.6.1 -DCMAKE_CXX_COMPILER=mpicxx \
      -DGeant4_DIR=/home/gfnun/Geant4-10.6/install/lib/Geant4-10.6.1 /home/gfnun/Geant4-10.6/src/geant4.10.06.p01/examples/extended/parallel/MPI/examples/exMPI01
$ sudo make
$ sudo make install
\end{lstlisting}

Ejecutar:

\begin{lstlisting}[language=bash,style=mystyle]
$ cd build
$ mpiexec -np 4 ./exMPI01
\end{lstlisting}

\subsection{Ejemplo exMPI04}

Compilar, para compilar se debe hacer desde el compilador de open mpi instalado en la carpeta sharedFolder/

\begin{lstlisting}[language=bash,style=mystyle]
$ mkdir build
$ cd build
$ sudo cmake -DG4mpi_DIR=/home/gfnun/Geant4-10.6/geant4-mpi/lib/G4mpi-10.6.1 -DCMAKE_CXX_COMPILER=/home/gfnun/sharedFolder/openmpi-install/bin/mpicxx \
      -DGeant4_DIR=/home/gfnun/Geant4-10.6/install/lib/Geant4-10.6.1 /home/gfnun/Geant4-10.6/src/geant4.10.06.p01/examples/extended/parallel/MPI/examples/exMPI04
$ make
$ make install
\end{lstlisting}

Ejecutar con Open mpi:

Exportar variables, en este caso la instalación de open mpi se realizó en sharedFolder:

\begin{lstlisting}[language=bash,style=mystyle]
export PATH=/home/gfnun/sharedFolder/openmpi-install/bin:$PATH
export LD_LIBRARY_PATH=/home/gfnun/sharedFolder/openmpi-install/lib:$LD_LIBRARY_PATH 
\end{lstlisting}

De esta manera se ejecuta en la máquina local:

\begin{lstlisting}[language=bash,style=mystyle]
$ cd build
$ mpiexec -n 4 ./exMPI04
\end{lstlisting}




% \begin{itemize}

%     \item \textbf{Usando MPICH:}
    
% Nos aseguramos de usar MPICH

% \begin{lstlisting}[language=bash,style=mystyle]     
% export PATH=/home/gfnun/sharedFolder/mpich-install/bin:$PATH
% export LD_LIBRARY_PATH=/home/gfnun/sharedFolder/mpich-install/lib:$LD_LIBRARY_PATH 
% \end{lstlisting}     

% Verificamos:

% \begin{lstlisting}[language=bash,style=mystyle]     
% which mpiexec
% \end{lstlisting} 

% Y debe arrojarnos el directorio desde donde se está ejecutando esta instrucción en este caso el directorio donde esta instalado MPICH.
% \\
% Creamos el archivo en dónde especificamos como usar los procesadores. Lo llamaremos \textbf{hostfile}, aquí le indicamos usar 1 procesador en el master, 4 para el escalvo 1 y 5 para el esclavo2.

% \begin{lstlisting}[language=bash,style=mystyle]     
% master:1
% slave1:4
% slave2:5
% \end{lstlisting}  

% En consola ejecutamos el siguiente comando:
% \begin{lstlisting}[language=bash,style=mystyle]     
% mpirun -f hostfile -n 10 ./exMPI01
% \end{lstlisting} 

% Si se desea especificar:
% \begin{lstlisting}[language=bash,style=mystyle]     
% mpirun -hosts slave1,slave2^Cn 2 ./exMPI01
% \end{lstlisting} 



%     \item \textbf{Usando open-MPI:}
    
% Creamos el archivo en dónde especificamos como usar los procesadores. Lo llamaremos \textbf{hostfile}, aquí le indicamos usar 1 procesador en el master, 4 para el escalvo 1 y 5 para el esclavo2.

% \begin{lstlisting}[language=bash,style=mystyle]     
% master slots=1
% slave1 slots=2
% slave2 slots=2
% \end{lstlisting}     
    
% \end{itemize}

\newpage





