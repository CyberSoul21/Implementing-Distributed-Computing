\section{Instalación de Geant4}
\subsection{Instalación de prerrequisitos}

\begin{itemize}
    \item \textbf{CLHEP}
    \url{http://proj-clhep.web.cern.ch/proj-clhep/clhep23.html} \\
leer el archivo README.txt luego crear una carpeta Build y una install dentro de la carpeta build hacer:

\begin{lstlisting}[language=bash,style=mystyle]
cmake -DCMAKE_INSTALL_PREFIX=<install_dir> <source_code_dir>
cmake --build . --config RelWithDebInfo
cmake --build . --target install
\end{lstlisting} 




    \item \textbf{Zlib}%%%%%%%%%%%%%%%%%%%%%%%%%%%%%%%%%%%%
    
\begin{lstlisting}[language=bash,style=mystyle]
sudo apt install zlib1g-dev 
\end{lstlisting}     
    
    
    \item \textbf{QT}%%%%%%%%%%%%%%%%%%%%%%%%%%%%%%%%%%%%%%%%
    
\begin{lstlisting}[language=bash,style=mystyle]
sudo apt install qt4-default
\end{lstlisting}      
    
    
    \item \textbf{OpenGL}%%%%%%%%%%%%%%%%%%%%%%%%%%%%%%%%%%%%%
    
\begin{lstlisting}[language=bash,style=mystyle]
sudo apt-get install libglu1-mesa-dev freeglut3-dev mesa-common-dev
\end{lstlisting}      
    
    
    \item \textbf{ray tracer}%%%%%%%%%%%%%%%%%555
    
\begin{lstlisting}[language=bash,style=mystyle]
sudo apt install povray
\end{lstlisting}     
    
    
    \item \textbf{X11 libraries}%%%%%%%%%%%%%%%%%%%%%%%%%%%%%%%%%%%%
    
\begin{lstlisting}[language=bash,style=mystyle]
sudo apt-get install xorg-dev
\end{lstlisting}  


    \item \textbf{X11 libraries}%%%%%%%%%%%%%%%%%%%%%%%%%
    
\begin{lstlisting}[language=bash,style=mystyle]
sudo apt-get install xorg-dev
\end{lstlisting} 


    \item \textbf{g++ y gcc}%%%%%%%%%%%%%%%%%%%%%%%%%%%%%%%%%
    
\begin{lstlisting}[language=bash,style=mystyle]
sudo apt-get update -y && \

sudo apt-get install -y wget g++-5 gcc-5 cmake libexpat1-dev vim cmake-curses-gui freeglut3 freeglut3-dev mesa-utils python libx11-dev libxmu-dev expat && \

sudo apt-get install -y libfontconfig1-dev libfreetype6-dev libxcursor-dev libxext-dev libxfixes-dev libxft-dev libxi-dev libxrandr-dev libxrender-dev && \

sudo apt-get install libssl-dev
\end{lstlisting} 


\end{itemize}{}


\newpage

\subsection{Instalación de Geant4}

\begin{enumerate}
    \item Creación de carpetas donde estará alojada la instalación
    
\begin{lstlisting}[language=bash,style=mystyle]

$ cd $HOME
$ sudo mkdir -p $HOME/Geant4-10.6/src 
$ sudo mkdir -p $HOME/Geant4-10.6/build 
$ sudo mkdir -p $HOME/Geant4-10.6/install 
$ sudo mkdir -p $HOME/Geant4-10.6/data 

\end{lstlisting}
    
    \item Descarga de Geant4:
    
\begin{lstlisting}[language=bash,style=mystyle]

$ G4_VERSION="10.06.p01"
$ QT_VERSION="4.8.7"
$ sudo wget http://cern.ch/geant4-data/releases/geant4.${G4_VERSION}.tar.gz
$ sudo tar xf geant4.${G4_VERSION}.tar.gz -C $HOME/Geant4-10.6/src
$ sudo rm geant4.${G4_VERSION}.tar.gz 

\end{lstlisting}     


    \item Compilación:
    
\begin{lstlisting}[language=bash,style=mystyle]

$ cd $HOME/Geant4-10.6/build
$ cmake -DCMAKE_INSTALL_PREFIX=$HOME/Geant4-10.6/install \
      -DGEANT4_USE_OPENGL_X11=ON \
      -DQT_QMAKE_EXECUTABLE=/usr/local/Trolltech/Qt-${QT_VERSION}/bin/qmake \
      -DGEANT4_USE_QT=OFF \
      -DGEANT4_INSTALL_DATA=ON \
      -DGEANT4_INSTALL_DATADIR=$HOME/Geant4-10.6/data \
      -DGEANT4_BUILD_MULTITHREADED=ON \
      -DGEANT4_INSTALL_EXAMPLES=ON \
      ../src/geant4.${G4_VERSION}
      
$ make -j`nproc` 

\end{lstlisting}     

    
    \item Instalación:

\begin{lstlisting}[language=bash,style=mystyle]

$ sudo make install
 
\end{lstlisting}     

    \item Cargar varibles de entorno:

\begin{lstlisting}[language=bash,style=mystyle]

$ echo "export LD_LIBRARY_PATH=${LD_LIBRARY_PATH}:/usr/local/Trolltech/Qt-${QT_VERSION}/lib" >> $HOME/.bashrc
$ echo "source $HOME/Geant4-10.6/install/bin/geant4.sh" >> $HOME/.bashrc


\end{lstlisting}     


    \item Verificar instalación y configuración


\begin{lstlisting}[language=bash,style=mystyle]

$ ./geant4-config --help
 
\end{lstlisting} 

\item Por último crear variables de entorno:


\begin{lstlisting}[language=bash,style=mystyle]
echo "source $HOME/geant4/install/bin/geant4.sh" >> $HOME/.bashrc
\end{lstlisting} 

\end{enumerate}

\newpage

\subsubsection{Configurar variables de entorno para Geant4}

\begin{enumerate}
    \item Abrir el archivo bashrc:
    
\begin{lstlisting}[language=bash,style=mystyle]

$ sudo nano ~/.bashrc

\end{lstlisting} 

    \item Añadir al final las siguiente líneas:
    
\begin{lstlisting}[language=bash,style=mystyle]

export G4ABLADATA=/home/gfnun/Geant4-10.6/install/share/Geant4-10.6.1/data/G4ABLA3.1
export G4ENSDFSTATEDATA=/home/gfnun/Geant4-10.6/install/share/Geant4-10.6.1/data/G4ENSDFSTATE2.2
export G4INCLDATA=/home/gfnun/Geant4-10.6/install/share/Geant4-10.6.1/data/G4INCL1.0
export G4LEDATA=/home/gfnun/Geant4-10.6/install/share/Geant4-10.6.1/data/G4EMLOW7.9.1
export G4LEVELGAMMADATA=/home/gfnun/Geant4-10.6/install/share/Geant4-10.6.1/data/PhotonEvaporation5.5
export G4NEUTRONHPDATA=/home/gfnun/Geant4-10.6/install/share/Geant4-10.6.1/data/G4NDL4.6
export G4PARTICLEXSDATA=/home/gfnun/Geant4-10.6/install/share/Geant4-10.6.1/data/G4PARTICLEXS2.1
export G4PIIDATA=/home/gfnun/Geant4-10.6/install/share/Geant4-10.6.1/data/G4PII1.3
export G4REALSURFACEDATA=/home/gfnun/Geant4-10.6/install/share/Geant4-10.6.1/data/RealSurface2.1.1
export G4SAIDXSDATA=/home/gfnun/Geant4-10.6/install/share/Geant4-10.6.1/data/G4SAIDDATA2.0

\end{lstlisting}
\end{enumerate}


\subsubsection{Ejecutar Ejemplo B1}

\begin{enumerate}

    \item Ir a la carpeta del código fuente del ejemplo y crear una carpeta para la compilación.
    
\begin{lstlisting}[language=bash,style=mystyle]

$ cd /home/gfnun/Geant4-10.6/src/geant4.10.06.p01/examples/basic/B1
$ mkdir build-B1 

\end{lstlisting}

    \item Compilar y ejecutar:
    
\begin{lstlisting}[language=bash,style=mystyle]

$ cmake –DGeant4_DIR=/home/gfnun/Geant4-10.6/install/lib/Geant4-10.6.1 /home/gfnun/Geant4-10.6/src/geant4.10.06.p01/examples/basic/B1
$ make -j4
$ ./exampleB1

\end{lstlisting}  

\end{enumerate}

\newpage










    
    
