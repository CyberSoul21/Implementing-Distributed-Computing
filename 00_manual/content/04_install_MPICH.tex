\section{Instalación y configuración de MPICH}

\subsection{Instalando MPICH2}

\begin{itemize}


    \item Descargar MPICH \url{http://www.mpich.org/downloads/}
    

    \item Descomprimir
\begin{lstlisting}[language=bash,style=mystyle]     
tar -xzf mpich-3.3.2.tar.gz 
\end{lstlisting}   

    \item Compilar e instalar 
\begin{lstlisting}[language=bash,style=mystyle]    
cd mpich-3.3.2
sudo mkdir build
cd build
sudo mkdir ../mpich-install
  
sudo ../configure --prefix=/home/gfnun/mpich-install/ --disable-fc --disable-f77 --enable-shared
\end{lstlisting} 


\item Aparece el siguiente error:

\begin{lstlisting}[language=bash,style=mystyle2]  
configure: error: No Fortran 77 compiler found. If you dont need to build any Fortran programs, you can disable Fortran support using --disable-fortran. If you do want to build Fortran programs, you need to install a Fortran compiler such as gfortran or ifort before you can proceed.
\end{lstlisting} 



\item Corrección error, debido a que no está instalado Fortran, para este caso no se necesita, por ende se coloca la bandera de desactivación.

\begin{lstlisting}[language=bash,style=mystyle]    
sudo ../configure --prefix=/home/gfnun/mpich-install/ --disable-fc --disable-f77 --enable-shared --disable-fortran 
\end{lstlisting} 



\item Aparece el mensaje:

\begin{lstlisting}[language=bash,style=mystyle2]    
Configuration completed.
\end{lstlisting} 



\item Compilación e instalación.

\begin{lstlisting}[language=bash,style=mystyle]      
sudo make
sudo make install
\end{lstlisting} 


\item Hacer la siguiente exportación de las variables de entorno con las que se va a trabajar.

\begin{lstlisting}[language=bash,style=mystyle]     
export PATH=/home/gfnun/sharedFolder/mpich-install/bin:$PATH
export LD_LIBRARY_PATH=/home/gfnun/sharedFolder/mpich-install/lib:$LD_LIBRARY_PATH 
\end{lstlisting} 



\end{itemize}